\documentclass{standalone}

\begin{document}
  Before concluding we would like to have a brief discussion regarding complexity theory and the simplex method. Over the years the discussion on whether there exists a pivoting strategy such that the simplex method becomes a strongly polynomial time algorithm has been significant. Before the Klee-Minty cube \cite{klee1972good} the worst case complexity of the simplex method was not known. Following the demonstration that the simplex method with Dantzig's rule runs at worst in exponential time considerable research has been done into alternative pivoting strategies. As outlined in the introduction (section \ref{sec:intro}) and section \ref{sec:pivot} each new proposed pivoting strategy has been proven to either have an exponential worst-case run-time, a sub-exponential lower bound or an unknown time bound which means that they cannot be used to obtain a strongly polynomial time algorithm. Even the randomized schemes that have been harder to reason about have been proven to have exponential \cite{broder1995worst} or sub-exponential \cite{goldwasser1995survey} running times. The search continues to this day and a proof for whether there exists a pivoting rule such that simplex algorithm is definitely a strongly polynomial algorithm is an open question. One interesting development is \cite{adler2014simplex} where the authors show for some of the popular simplex pivoting rules, if a pivoting rule is PSPACE-complete then that pivoting rule cannot be used to obtain a strongly polynomial simplex method unless P $=$ PSPACE. Of course, the question of P $\overset{?}{=}$ PSPACE is also an important open problem. This work seems like a right step in defining an intractability proof for a general class of simplex pivoting rules \cite{adler2014simplex}.\par
  On the other hand, we know from the smoothed analysis introduced in \cite{spielman2004smoothed} that the simplex method runs in polynomial time for most real-world problems. What is interesting is that it does not seem like most industrial solvers or algorithms included numerical computing software utilize randomization for pivoting. The most popular choice appears to be the primal-dual simplex algorithm with some deterministic pivoting choice that is known to work well due to years of testing and knowledge. Our experiments show that randomized strategies could be promising in obtaining significant speed-ups in solving a linear program but there are numerical instability issues that need to be addressed. Furthermore, more specialized algorithms such as Clarkson's first algorithm could prove to be effective for problems with a specific dimensionality (i.e. where the number of contraints are much greater than the number of variables). We propose that more testing needs to be done where the tests address the issues raised in section \ref{sec:eval} and the scope of the comparisons should be widened to primal-dual methods. Yet, it is clear from this project that randomized algorithms can be potentially more effective alternatives to the current deterministic strategies used to solve linear programs and further experimental work should be done.
\end{document}