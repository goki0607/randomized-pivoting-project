\documentclass{standalone}

\begin{document}

The simplex method for solving linear programming problems is often considered as one of the most important algorithms developed in the 20th century \cite{cipra2000best}. A simplex or simplex-like method for finding the solution to a linear program is incomplete without a pivoting rule or strategy being specified. The primary goal of pivoting is to define a new search direction for the simplex method which in turn defines a new vertex with a lower value of the objective function. In addition to finding an optimal vertex, we would also like our pivoting rule to get there as fast as possible and not be effected by stalling and/or cycling in the case of degenaracy. These goals can sometimes be in conflict with each other as fast rules such as Dantzig's least index rule can find a solution quickly for most problems but does not guarantee termination due to cycling. On the other hand, Bland's least index rules as introduced in \cite{bland1977new} are guaranteed to terminate but are considerably slower in practice. Thus, a choice must be made which greatly influences the observed performance when solving a linear program using the simplex method. Interestingly, the choice of a pivot strategy also plays an important role in the theoretical run-time properties of simplex and simplex-like methods.\par 
The theoretical run-time of a simplex or simplex-like method is primarily determined by its pivoting rule. Since the other aspects of the simplex method, including solving a system of equations and checking for optimality, can be done in polynomial time, the pivoting rule will determine which vertices will be visited and in which order they will be visited in. This in turn determines the number of iterations needed before finding a solution to a feasible and bounded linear program. The process of selecting a pivot in general runs in polynomial time. However, the actual path a pivoting strategy leads to can cause the simplex method to run in exponential, and sometimes sub-exponential, time for a specific linear program. The question of whether the simplex method is a strongly polynomial algorithm is still an open question and this is because all currently known pivoting rules lead to at worst exponential time complexity \cite{adler2014simplex}. However, we also do not currently have a proof showing that the simplex method cannot be strongly-polynomial for all pivoting methods. Beginning with the construction of the Klee-Minty cube to show that Dantzig's rule leads to exponential time performance of the simplex method \cite{klee1972good}, for each new proposed pivoting algorithm it has either been proven that the rule has exponential time performance in the worst case or its worst-case complexity is not yet known. Most of these proofs are done by creating a specific polytope or problem where the pivoting rule exhibits the desired worst-case behavior.\par
This search for a pivoting rule that leads to a polynomial time simplex instance has sparked an interest in randomized pivoting strategies. Recent advances include the development of randomized simplex-like algorithms that have sub-exponential worst-case upper bounds. A nice survey of these rules can be found in \cite{goldwasser1995survey}. Due to the analysis of Spielman and Teng we also know that the simplex method for deterministic pivoting rules, specifically for the shadow vertex method, has polynomial smoothed complexity \cite{spielman2004smoothed}. That is, this work provides a theoretical explanation to why the deterministic pivoting rules in practice exhibit polynomial time behavior for most problems. No such work exists for randomized simplex methods and furthermore, no known major experimental evaluations or implementations of randomization exist. Overall, research into randomized simplex methods shows certain benefical theoretical properties but experimental studies are hard to come by.\par
This project aims to explore and experimentally evaluate two randomized simplex algorithms and one simplex-like algorithm and compare them against three well known and widely used deterministic simplex pivoting rules. The main goal is to observe numerical issues and the behavior of the randomized methods in a practical setting.\par
We consider the random edge and random facet rules alongside a modified version of Clarkson's first algorithm from \cite{clarkson1995vegas} as our randomized implementations. Our determinsitic strategies are Dantzig's rule, Bland's rules and the steepest edge rule. Section \ref{sec:lp} briefly goes over linear programming basics and concepts. Then we move onto section \ref{sec:pivot} where we describe the three deterministic pivoting rules and the three random pivoting strategies that will be experimentally evaluated. Section \ref{sec:implementation} goes over the Matlab scripts implemented for the project. Then, section \ref{sec:experiments} presents the experiments run and the results obtained where \ref{tab:summary} is a table that attempts to summarize the results. Here an analysis of the pivoting rules is also given. In section \ref{sec:eval} the flaws in the analysis are outlined along. Finally, section \ref{sec:conc} concludes the paper. A GitHub repository containing all the material used for this project can be found here: \url{https://github.com/goki0607/randomized-pivoting-project}.

\end{document}